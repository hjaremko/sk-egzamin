\documentclass[../sk-egzamin.tex]{subfiles}

\begin{document}

\question{
Proszę porównać protokół TCP i UDP.
}

\question{
 Proszę opisać (podać przykład) jak może dojść do pętli routowania w protokole
RIP (pętla ma obejmować dwa sąsiednie routery albo więcej niż dwa routery).
}

\question{
Proszę opisać działanie protokołu EIGRP.
}

\question{
Co to jest multiemisja \textit{(multicast)}? Co to jest IGMP?
Jakie jest wsparcie dla przesyłania grupowego w technologii Ethernet?
}

\question{
Proszę opisać protokół IPSec.
}

\question{
Proszę opisać mechanizmy w protokole IP związane z mobilnością.
}

\question{
Proszę opisać protokół IPv6.
}

\question{
Proszę opisać, jak działa mechanizm 6to4.
}

\question{
Do czego służy protokół STP i jak działa \textit{(podstawowe zasady)}.
Jak zbudować prostą lokalną sieć komputerową na przełącznikach z wykorzystaniem
warstwy dostępu \textit{(access)}, warstwy dystrybucji \textit{(distribution)}
i warstwy rdzenia \textit{(core)}?
}

\question{
Co to są VLANy i jak można je konfigurować?
}

\question{
Proszę podać podstawowe zasady pisania programów z wykorzystaniem
interfejsu gniazd. Jak działa serwer iteracyjny, a jak serwer współbieżny
\textit{(działający na procesach)}? Pytanie obejmuje zagadnienia w zakresie
przedstawionym na ostatnim wykładzie.
}

\question{
Proszę krótko scharakteryzować protokół BGP \textit{(w zakresie omówionym na
wykładzie)}.
}

\question{
Adres \texttt{192.168.7.0} z maską \texttt{255.255.252.0} to adres
\begin{enumerate}[label=\alph*)]
    \item hosta
    \item poprawny, ale maska jest niepoprawna
    \item sieci
    \item rozgłoszenia w podsieci lokalnej
    \item niepoprawny
\end{enumerate}
}

\question{
Jeśli adres IP hosta jest \texttt{192.168.14.0} z maską \texttt{255.255.248.0},
to adres sieci jest następujący:
\begin{enumerate}[label=\alph*)]
    \item \texttt{192.168.7.0}
    \item \texttt{192.168.14.0}
    \item \texttt{192.168.8.0}
    \item \texttt{192.168.1.0}
    \item \texttt{192.168.248.0}
\end{enumerate}
}

\question{
Adres \texttt{255.255.255.255} oraz \texttt{152.16.255.255} to adresy
odpowiednio:
\begin{enumerate}[label=\alph*)]
    \item obydwa są niepoprawne
    \item ograniczonego rozgłoszenia i rozgłoszenia w sieci \texttt{152.16.0.0}
    \item pierwszy jest niepoprawny, drugi to rozgłoszenie w sieci lokalnej
    \item drugi jest niepoprawny a pierwszy to rozgłoszenie w sieci lokalnej
    \item pierwszy to adres domyślnej trasy w routerze, drugi to rozgłoszenie w
    sieci lokalnej
\end{enumerate}
}

\question{
Który z poniższych adresów hosta jest z podsieci, w której można zaadresować
62 komputery?
\begin{enumerate}[label=\alph*)]
    \item \texttt{146.5.30.72/26}
    \item \texttt{182.43.23.0/30}
    \item \texttt{163.32.10.64/62}
    \item \texttt{132.43.23.33/16}
    \item \texttt{133.34.54.255/32}
\end{enumerate}
}


\pagebreak
\end{document}
