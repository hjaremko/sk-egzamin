\documentclass[../sk-egzamin.tex]{subfiles}

\begin{document}

\question{
Proszę porównać protokół TCP i UDP.
}

\question{
 Proszę opisać (podać przykład) jak może dojść do pętli routowania w protokole
RIP (pętla ma obejmować dwa sąsiednie routery albo więcej niż dwa routery).
}

\question{
Proszę opisać działanie protokołu EIGRP.
}

\question{
Co to jest multiemisja \textit{(multicast)}? Co to jest IGMP?
Jakie jest wsparcie dla przesyłania grupowego w technologii Ethernet?
}

\question{
Proszę opisać mechanizmy w protokole IP związane z mobilnością.
}

\question{
Proszę opisać protokół IPv6.
}

\question{
Proszę opisać, jak działa mechanizm 6to4.
}

\question{
Do czego służy protokół STP i jak działa \textit{(podstawowe zasady)}.
Jak zbudować prostą lokalną sieć komputerową na przełącznikach z wykorzystaniem
warstwy dostępu \textit{(access)}, warstwy dystrybucji \textit{(distribution)}
i warstwy rdzenia \textit{(core)}?
}

\question{
Proszę opisać jak sprawdzić, czy połączenie ze stroną WWW banku jest
bezpieczne.
}

\question{
Proszę podać podstawowe zasady pisania programów z wykorzystaniem
interfejsu gniazd. Jak działa serwer iteracyjny, a jak serwer współbieżny
\textit{(działający na procesach)}? Pytanie obejmuje zagadnienia w zakresie
przedstawionym na ostatnim wykładzie.
}

\question{
Proszę krótko scharakteryzować protokół BGP \textit{(w zakresie omówionym na
wykładzie)}.
}


\pagebreak
\end{document}
