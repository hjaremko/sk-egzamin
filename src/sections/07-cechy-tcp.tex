\documentclass[../sk-egzamin.tex]{subfiles}

\begin{document}

\question{
Jakie są podstawowe cechy protokołu TCP?
}

\subsection*{Ważne cechy TCP}
\begin{itemize}
    \item Partnerzy \parit{komunikujące się procesy} tworzą połączenie
    z wykorzystaniem mechanizmu uzgodnienia \parit{uzgadnianie trójfazowe
    - three-way handshake}.

    \item Zamknięcie połączenia odbywa się z wykorzystaniem mechanizmu
    uzgodnienia, podczas którego partnerzy wyrażają zgodę na zamkniecie
    połączenia.

    \item TCP zapewnia \textbf{sterowanie przepływem}.
    Informuje partnera o tym ile bajtów ze strumienia danych może od niego
    przyjąć \parit{\textbf{okno oferowane} - advertized window}.

    Rozmiar ten \textbf{zmienia się dynamicznie} i
    jest równy rozmiarowi wolnego miejsca w buforze odbiorcy.
    Zero oznacza, że nadawca musi zaczekać, aż program użytkowy odbierze dane
    z bufora.

    \item Dane ze strumienia są dzielone na fragmenty, które według TCP mają
    najlepszy do przesłania rozmiar.
    Jednostka przesyłania danych nazywa się \textbf{segmentem}.

    \item Zapewnia \textbf{niezawodność połączenia}.
\end{itemize}

\pagebreak
\end{document}
