\documentclass[../sk-egzamin.tex]{subfiles}

\begin{document}

\question{
Proszę opisać działanie protokołu OSPF.
}

\subsection*{Cechy}
\begin{itemize}
    \item Protokół \textbf{stanu łącza}.
    \begin{itemize}
        \item Łącze reprezentuje połączenie z sąsiednimi routerami, komputerem
        lub siecią.

        \item \textbf{Stan łącza} składa się z wielu atrybutów np.
        prędkość transmisji, opóźnienie, adres IP, sąsiedzi.
    \end{itemize}

    \item Protokół \textbf{bezklasowy}.
    \item Metryką jest \textbf{szybkość łącza}.
    \item Umożliwia \textbf{uwierzytelnienie}.
    \item \textbf{Szybko uzyskuje zbieżność} \parit{trigger updates}.
    \item \textbf{Flooding} do utworzenia map topologii.
    Generuje bardzo duży ruch w sieci.
    \item Ruch w wielodostępnych fragmentach sieci
    (więcej niż dwa routery mogą być bezpośrednio połączone np. Ethernet)
    jest ograniczany poprzez wybranie tzw. DR (designated router), z którym
    (jednym) wymieniają informacje wszystkie inne routery.
    Wybierany jest również BDR (backup designated router).
    \item Trudność w konfiguracji.
\end{itemize}

\subsection*{Działanie}
\begin{itemize}
    \item Każdy router buduje informację o połączeniach w całej sieci
    \parit{systemu autonomicznego} w formie grafu.
    \item Wyliczane są najkrótsze ścieżki od danego węzła do wszystkich innych
    węzłów za pomocą \textbf{algorytmu Dijkstry}.
    \item Nie są przesyłane okresowo tablice routowania.
    \item Przesyłane są jedynie informacje o zmianach w stanie łączy
    \textbf{do wszystkich węzłów}.
\end{itemize}

\subsection*{Obszary}
Obliczanie najkrótszych ścieżek w grafie stosunkowo mocno obciąża procesor i
pamięć routera. Dzieląc sieć na obszary zmniejszamy wielkość grafu.

\begin{itemize}
    \item Każdy obszar OSPF ma przypisany numer. Jeśi jest tylko jeden obszar,
    to musi mieć numer 0.
    \item Jeśli obszarów jest więcej, to wszystkie muszą się łączyć poprzez
    obszar 0 \parit{najlepiej fizycznie, można wirtualnie}.
\end{itemize}

\subsubsection*{Rodzaje obszarów}
\begin{itemize}
    \item \textbf{Normalne}
    \item \textbf{Stub Area} - NIE są wprowadzane trasy
    zewnętrzne \parit{z innych protokołów routingu},
    natomiast sumy tras z innych obszarów są.
    \item \textbf{Totally Stubby Area} - nie są wprowadzane ani trasy
    zewnętrzne, ani sumy tras z innych obszarów.
    Wyjście z takiego obszaru tylko przez trasę domyślną.
    \item \textbf{Not So Stubby Area} \parit{NSSA} - obszar Stub, do którego
    wprowadzane są pewne \parit{na ogól nieliczne} trasy zewnętrzne, które
    następnie przekazywane są do innych obszarów tak jak sumy tras.
    \item \textbf{Not So Stubby Totally Stobby Area} - połączenie NSSA i
    Totally Stubby Area.
\end{itemize}
Routery na granicach obszarów powinny być odpowiednio skonfigurowane.

\subsection*{Nagłówek OSPF}
\begin{itemize}
    \item Typ pakietu, ID routera, ID obszaru
\end{itemize}

\subsection*{Pakiety OSPF}
\begin{itemize}
    \item \textbf{HELLO} - nawiązywanie sąsiedztwa, sprawdzanie obecności
    sąsiadów i stanu łącza.
    \item \textbf{DBD} - skrócona lista bazy danych łącze stan.
    \item \textbf{LSR} - żądanie dodadtkowych informacji o wpisie z DBD.
    \item \textbf{LSU} - aktualizacja będąca odpowiedzą na LSR.
    \item \textbf{LSA} - potwierdzenie odebrania LSU.
\end{itemize}


\pagebreak
\end{document}
