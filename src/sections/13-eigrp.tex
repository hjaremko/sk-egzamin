\documentclass[../sk-egzamin.tex]{subfiles}

\begin{document}

\question{
Proszę opisać działanie protokołu EIGRP.
}

\subsection*{EIGRP}
\begin{itemize}
    \item Obsuguje \textbf{adresowanie bezklasowe.}
    \item Numer systemu autonomicznego taki sam jak w komunikujących się
    routerach.
    \item \textbf{IGRP i EIGRP} mogą ze sobą współpracować, jeśli mają ten sam
    numer.
    \begin{itemize}
        \item Nastąpi przeliczenie metryki.
        \item Trasa z IGRP jest traktowana jak trasa zewnętrzna.
    \end{itemize}
    \item Metryka 32 bitowa.
    \begin{itemize}
        \item Aktualizacje zawierają liczbę skokó dla trasy, jednak liczba
        skoków nie jest brana pod uwagę przy wyliczaniu metryki.
    \end{itemize}
\end{itemize}

\subsection*{Kluczowe technologie i idee}
\begin{itemize}
    \item Wykrywanie \textbf{sąsiadów}.
    \item Diffusing Update Algorithm \textbf{DUAL}.
    \item \textbf{Trigger updates} po wykryciu zmiany lub nowego sąsiada.
    \item Komunikaty \textbf{HELLO}.
    \item Wyznaczają \textbf{succesorów, feasible succesorów}.
\end{itemize}

\subsection*{Zalety}
\begin{itemize}
    \item \textbf{Minimalne zużycie szerokości pasma} gdy sieć jest stabilna.
    \begin{itemize}
        \item W czasie normalnego stabilnego działania sieci jedynymi
        wymienianymi pakietamy są pakiety \textbf{HELLO}.
        \item Wydajne wykorzystanie szerokości pasma w czasie uzyskiwania
        zbieżności.
        \begin{itemize}
            \item Propagowane są jedynie zmiany, nie całe wektory odległości.
            \item Po wykryciu sąsiada uaktualnienie wysyłane jest tylko do
            niego (unicast).
        \end{itemize}
    \end{itemize}
    \item \textbf{Szybka zbieżność}.
\end{itemize}

EIGRP wykorzystuje specjalny \textbf{niezawodny protokół} w warstwie transportu
\parit{RTP - Reliable Transport Protocol}.

\begin{itemize}
    \item Aktualizacje są przesyłane na adres grupowy \texttt{224.0.0.10}.
    Potwierdzenia są przesyłane na adres unicast.
    Jeśli potwierdzenie z określonym numerem sekwencji nie nadejdzie w czasie
    RTO \parit{Retransmission Time Out}, pakiet z aktualizacją jest
    transmitowany, tym razem na adres jednostkowy.

    \item Zwykłe pakiety HELLO oraz potwierdzenia nie są potwierdzane.

    \item DUAL jest używany do wyznaczania sukcesorów i wykonalnych sukcesorów
    określających \textbf{trasy zapasowe}.

    \item Mechanizm wyznaczania tras zapasowych zapewna, że
    \textbf{nie ma w nich pętli routowania}.
\end{itemize}


\pagebreak
\end{document}
