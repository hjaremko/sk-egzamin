\documentclass[../sk-egzamin.tex]{subfiles}

\begin{document}

\question{
Jak działa ARP?
}

\textbf{ARP \parit{Address Resolution Protocol}} tłumaczy adresy między
warstwą internetową a warstwą interfejsu sieciowego.

\begin{itemize}
    \item Dotyczy tylko \textbf{IPv4}, w wersji IPv6 \textbf{nie jest}
    wykorzystywany, zastępują go inne mechanizmy.

    \item Ze względu na możliwość wymiany karty sieciowej w komputerze o
    określonym adresie IP jest dynamiczny.

    \item Oparty na metodzie rozgłoszeniowej i zasadzie żądania i odpowiedzi.
\end{itemize}

\subsection*{Działanie}
\begin{enumerate}
    \item Sprawdza w pamięci podręcznej tablicę przyporządkowującą adresy
    MAC adresom IP.

    \item Jeśli odpowiedni wpis \textbf{nie zostanie znaleziony} zostaje
    wysłana ramka \texttt{ARP Request Message}.
    Jest to ramka rozgłoszeniowa do wszystkich węzłów fizycznego segmentu sieci,
    do którego przyłączony jest nadawca \parit{ARP requester}.

    W ramce jest zapisane pytanie:
    \texttt{Kto ma adres IP 149.159.12.15?}\\
    \texttt{Proszę podać mi adres MAC}.

    \item Po otrzymaniu \texttt{ARP Request} uaktualniane są również cache ARP
    w komputerach, które otrzymały tę ramkę, a któe miały w cache IP komputera,
    który wysłał żądanie.

    \item Wpisy w cache ARP są usuwane po okresie nieużywania rzędu kilku minut.
\end{enumerate}

\subsection*{Możliwe są dwa przypadki:}
\begin{enumerate}
    \item \textbf{Węzeł docelowy znajduje się w tym samym segmencie sieci}
    \begin{enumerate}
        \item \textit{ARP requester} pyta wprost kto ma docelowy IP.
        \item \textit{ARP responder} wysyła ARP Reply pod adres MAC, z któego
        przyszło żądanie.
        Adres ten jest znany, gdyż znajduje się w ramce zapytania ARP.
        \item Po wymianie ramek nadawca i odbiorca mają uaktualnione tablice
        w pamieci podręcznej.
    \end{enumerate}

    \item \textbf{Węzeł docelowy znajduje się w innym segmencie sieci}
    \begin{enumerate}
        \item Datagram musi być skierowany do domyślnego routera.
        \item \textit{ARP requester} pyta kto ma adres IP domyślnego routera.
        \item Router odpowiada wysyłająć ARP respond ze swoim adresem MAC.
        \item W przypadku, gdy w konfiguracji komputerów w sieci nie ma IP
        routera domyślnego, można skorzystać z \texttt{Proxy ARP}.
    \end{enumerate}
\end{enumerate}

\subsection*{Struktura ramki ARP}
\begin{itemize}
    \item \textbf{Typ sprzętu} \parit{2 oktety}
    \item \textbf{Typ protokołu} \parit{2 oktety}
    \item \textbf{Długość adresu sprzętu} \parit{1 oktet}
    \item \textbf{Długość adresu protokołu} \parit{1 oktet}
    \item \textbf{Kod operacji} \parit{2 oktety}
    \item \textbf{Adres sprzętu nadawcy} \parit{dla Ethernet 6 oktetów}
    \item \textbf{Adres protokołu nadawcy} \parit{dla IPv4 4 oktety}
    \item \textbf{Adres sprzętu docelowego} \parit{dla Ethernet 6 oktetów}
    \item \textbf{Adres protokołu docelowego} \parit{dla IPv4 4 oktety}
\end{itemize}

\subsection*{Wykrywanie zduplikowanych adresów IP \parit{zbędny ARP}}

\begin{itemize}
    \item Węzeł wysyła ARP Request z \textbf{zapytaniem o swój własny adres}.
    \begin{itemize}
        \item Jeśli ARP Reply \textbf{nie nadejdzie}, to znaczy, że w lokalnym
        segmencie \textbf{nie ma konfliktu}.

        \item Jeśli odpowiedź nadejdzie, oznacza to \textbf{konflikt}.
    \end{itemize}

    \item Węzeł już skonfigurowany traktowany jest jako węzeł z poprawnym
    adresem \parit{węzęł zgodny, defending node}.

    \item Węzeł wysyłający \textit{zbędny ARP} jest \textbf{węzłem konfliktowym}.

    \item Węzeł konfliktowy \textbf{wprowadza błąd} w cache ARP komputerów w
    \textbf{całym segmencie sieci}.
    ARP Reply z węzła zgodnego nie naprawia sytuacji
    \parit{ramka ARP nie jest ramką rozgłoszeniową}.
\end{itemize}

\subsection*{Proxy ARP}

Router ze skonfigurowanym mechanizmem Proxy ARP
\textbf{odpowiada na ramki ARP Request w imieniu wszystkich węzłów}
– komputerów spoza segmentu sieci lokalnej.
Może być używany jest np. w sytuacji, gdy komputery w sieci nie mają ustawionego
domyślnego routera.
Routery mogą mieć włączoną standardowo opcję Proxy ARP, wówczas jeśli jakiś
komputer wyśle ARP Request z adresem spoza danej sieci lokalnej (zwykle to nie
następuje), to router odpowie \textit{w imieniu} komputera zewnętrznego.

\pagebreak
\end{document}
