\documentclass[../sk-egzamin.tex]{subfiles}

\begin{document}

\question{
Co to jest multiemisja \textit{(multicast)}? Co to jest IGMP?
Jakie jest wsparcie dla przesyłania grupowego w technologii Ethernet?
}

\subsection*{Multiemisja - transmisja grupowa}

\begin{itemize}
    \item Wysyłanie jednego pakietu ze źródła do wielu miejsc docelowych.
    Pakiety są kopiowane w routerach i switchach warstwy drugiej.

    \item Mniejsze obciążenie sieci, wieksza skalowalność w stosunku do
    unicastu.

    \item Schematy jeden-do-wielu, wiele-do-wielu.

    \item Komunikaty w większości protokołów routowania mają zarezerwowane
    adresy multiemisji.

    \item Aby uczestniczyć w transmisji grupowaej, komputer musi sprawdzać
    określone adresy w przychodzących pakietach (IP) i generalnie w
    ramkach (MAC).

    \item Odbywa się z wykorzystaniem różnych mechanizmów i protokołów.
\end{itemize}

\subsection*{IGMP}
\begin{itemize}
    \item Wykorzystywany do dynamicznego rejestrowania/wyrejestrowania
    odbiornika w routerze.

    \item Komunikaty IGMP są przesyłane w pakietach IP z adresem docelowym
    typu multicast i ustawioną wartością TTL na 1.
\end{itemize}

\subsubsection*{IGMPv1}
\begin{itemize}
    \item \textbf{Membership query} wysyłany okresowo \parit{co kilkadziesiąt
    sekund} przez routery na wszystkie komputery.
    \item \textbf{Membership report} służy do zgłoszenia sie jako odbiorca
    pakietów wysyłanych na ten adres.
    \begin{itemize}
        \item Wysyłany też w odpowiedzi na membership query.
    \end{itemize}
\end{itemize}

Host po otrzymaniu membership query czeka pewien pseudolosowy czas i wysyła
membership report.
Jeśli w tym pseudolosowym czasie host usłyszy membership report od innego
hosta, to nie wysyła swojego raportu.

\textbf{Host 'po cichu' opuszcza grupę.}
Jeśli router nie dostanie raportu w odpowiedzi na \textbf{trzy kolejne}
membership query, router usuwa grupę z tablicy multicastu i przestaje
przesyłać pakiety kierowane do tej grupy.

\subsubsection*{IGMPv2}
\begin{itemize}
    \item Dodano \textbf{leave group}.
    \item Dodano \textbf{version2 membership report}.
    \item Membership query może być \textbf{group-specific query}.
\end{itemize}

\subsubsection*{IGMPv3}
\begin{itemize}
    \item Możliwość zgłaszania sie do grup z \textbf{wyspecifikowaniem
    adresu unicast} IPv4 pewnego nadawcy.
\end{itemize}

\subsection*{Wsparcie w Ethernet}
\begin{itemize}
    \item Ethernet daje możliwość adresowania \textbf{MAC typu multicast}.

    \item \textbf{23 bity adresu IPv4 są wprost wykorzystane w adresie MAC}.

    \item Każdy adres Ethernet multicast jest związany z 32 IPv4 klasy D
    \parit{różnica na 5 bitach}.
\end{itemize}

\pagebreak
\end{document}
