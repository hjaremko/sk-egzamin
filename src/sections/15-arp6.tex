\documentclass[../sk-egzamin.tex]{subfiles}

\begin{document}

\question{
Czym zastąpiono protokół ARP w IPv6?
}

Protokół ARP został zastąpiony przez protokół \texttt{ICMPv6}, który przejął
jego obowiązki.

\subsection*{ICMPv6}

\begin{itemize}
    \item Wykrywanie sąsiadów i szukanie routerów.
    \item Przekazywanie komunikatów o błędach w routowaniu.
    \item Komunikaty Echo Request i Echo Reply.
\end{itemize}

\subsubsection*{Wykrywanie sąsiada \parit{Neighbor Discovery}}

\begin{itemize}
    \item \textbf{Zastępuje ARP}, starą wersję ICMP router discovery oraz
    przekierowanie a także wykrywanie zduplikowanych adresów IP.
    \item Nowy mechanizm - wykrywanie czy sąsiad jest osiągalny.
    \begin{itemize}
        \item Uwalnia od kłopotów związanych z uszkodzeniem łączy,
        np. zdezaktualizowanymi pamięciami podręcznymi ARP.
        \item W przypadku uszkodzenia routera, automatycznie przekonfigurowuje
        się na połączenia na routery działające.
    \end{itemize}
\end{itemize}

Dla celów wykrywania adresów warstwy drugiej (analogicznie do ARP w IPv4) oraz
wykrywania osiągalności hosta i wykrywania powtórzenia adresów IP stosuje się
komunikaty Neighbor Solicitation oraz Neighbor Advertisement.
Komunikaty Neighbor Advertisement są wysyłane po zmianach w sieci oraz w
odpowiedzi na Neighbor Solicitation.
Neighbor Advertisement zawiera adres MAC.
Komunikaty Neighbor Solicitation i Router Solicitation są przesyłane na adres
IPv6 typu solicited-node multicast.

\pagebreak
\end{document}
