\documentclass[../sk-egzamin.tex]{subfiles}

\begin{document}

\question{
Co to jest routing statyczny i jak można go skonfigurować na routerach?
}

Routowanie jest to proces przesyłania pakietów (datagramów IP) od hosta
nadawczego do odbiorcy na ogół z wykorzystaniem routerów pośredniczących.

Każdy host oraz router podejmuje decyzje którędy przesłać datagram.
Decyzje te są podejmowane na podstawie tzw. \textbf{tabel routowania} oraz
pewnych reguł.

O \textbf{routingu statycznym} mówi się w przypadku, gdy w routerze tabela
routowania jest wypełniona wpisami statycznymi i nie zmienia się.

\subsection*{Tablica routowania IP}
\begin{itemize}
    \item Zawiera wpisy dotyczące tras do hostów, routerów i sieci.
    \item W hostach liczba wpisów jest dużo mniejsza od liczby wpisów w
    routerach i zawiera informacje o bramie domyślnej.
    Brama domyślna to router, do którego kierowany jest datagram, jeśli nie
    została znaleziona dla niego lepsza trasa w tablicy routowania.
\end{itemize}

\subsection*{Przykładowa struktura wpisu w tablicy routowania}
\begin{itemize}
    \item \textbf{Przeznaczenie} - ID sieci lub IP konkretnego
    hosta lub routera.
    Nie może zawierać jedynek w miejscu, gdzie w masce są zera \parit{binarnie}.
    \item \textbf{Maska sieci} - może też zawiera same jedynki.
    \item \textbf{Adres następnego skoku} - bez znaczenia dla połączeń
    punkt-punkt.
    \item \textbf{Interfejs} - przez który datagram będzie przesłany do
    następnego skoku.
    \item \textbf{Metryka}% - koszt trasy, im wyższa tym gorsza.
    - porównywane, gdy tablica zawiera więcej niż jedną trasę.
    %  do miejsca
    % przeznaczenia.
    \item \textbf{Odległość administracyjna} - statyczne wpisy mają wartość 1.
    \item \textbf{Trasa domyślna} - może być oznaczona przez \texttt{0.0.0.0}
    w polu przeznaczenie i \texttt{0.0.0.0} w polu maska sieci.
\end{itemize}

\subsection*{Konfiguracja}

\begin{minted}{bash}
R(config)# ip routing
R(config)# ip route 192.168.0.0 255.255.255.0 fa0/0
R(config)# ip route 192.168.0.0 255.255.255.0 10.0.0.1
\end{minted}

\pagebreak
\end{document}
