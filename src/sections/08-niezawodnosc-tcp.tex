\documentclass[../sk-egzamin.tex]{subfiles}

\begin{document}

\question{
Na czym polega niezawodność protokołu TCP?
}

\subsection*{Mechanizmy niezawodności}
\begin{itemize}
    \item \textbf{Potwierdzanie otrzymania segmentów z mechanizmem zegara.}
    \begin{itemize}
        \item Odebrany segment musi być potwierdzony przez odbiorcę przez
        wysłanie segmentu potwierdzajacego.

        \item Jeśli potwierdzenie nie nadejdzie w odpowiednim czasie,
        segment zostanie przesłany powtórnie.

        \item Czas oczekiwania \textbf{zmienia się dynamicznie} i zależy od
        stanu sieci \parit{obciązenia i parametrów}.
    \end{itemize}

    \item \textbf{Sumy kontrolne.}
    \begin{itemize}
        \item Jeśli segment zostanie nadesłany z niepoprawną sumą kontrolną,
        to \textbf{jest odrzucany} i nie jest przesyłane potwierdzenie.

        \item Nadawca po odczekaniu czasu RTO \parit{Retransmission Time Out}
        prześle utracony fragment jeszcze raz.
    \end{itemize}

    \item \textbf{Przywracanie kolejności nadchodzących segmentów.}
    \begin{itemize}
        \item Segmenty mogą nadchodzić w innej kolejności niż zostały wysłane.

        \item Oprogramowanie TCP przywraca prawidłową kolejność przed
        przekazaniem do aplikacji.
    \end{itemize}

    \item \textbf{Odrzucanie zdublowanych danych.}
\end{itemize}

\pagebreak
\end{document}
