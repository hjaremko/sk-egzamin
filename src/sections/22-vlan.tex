\documentclass[../sk-egzamin.tex]{subfiles}

\begin{document}

\question{
Co to są VLANy i jak można je konfigurować?
}

\subsection*{Wirtualne sieci LAN}
\begin{itemize}
    \item Sieć komputerowa wydzielona \textbf{logicznie} w ramach innej,
    większej sieci fizycznej.

    \item Tworzone przy pomocy \textit{switchy}.

    \item Przez konfigurację portów w switchach można tworzyć podsieci
    \textit{przełączalne}, ze wspólną dziedziną rozgłaszania w warstwie trzeciej
    \parit{i drugiej}.

    \item Każda sieć VLAN musi mieć przysany inny adres IP sieci.

    \item Są konfigurowane programowo, ewentualne zmiany konfiguracji nie
    wymagają zmian w okablowaniu.

    \item Komputery z każdej utworzonej sieci wirtualnej nie muszą być dołączone
    bezpośrednio do jednego przełącznika (jednak wszystkie połączone
    przełączniki muszą być odpowiednio skonfigurowane).

    \item Urządzenia należące do jednej sieci wirtualnej mogą się ze sobą
    komunikować tak jakby były w jednym segmencie sieci LAN
    (segment w warstwie drugiej i trzeciej ISO OSI).
    W całym takim segmencie na przykład działa protokół ARP, czyli ramka ARP
    request wysłana na adres rozgłoszeniowy MAC (48 jedynek) powinna dotrzeć
    (dociera) do wszystkich komputerów dołączonych do tej samej sieci VLAN
    (i przypiętych do tego samego przełącznika lub do różnych przełączników).

    \item \textbf{Komunikacja między sieciami wirtualnymi wymaga routera},
    nawet jeśli sieci wirtualne utworzone są tylko na jednym przełączniku.
    Są jednak przełączniki (przełączniki obsługujące warstwę 3), które potrafią
    wykonywać routowanie między równymi sieciami VLAN.
\end{itemize}

\subsubsection*{Mogą być budowane w oparciu o:}
\begin{itemize}
    \item \textbf{Porty w switchach.}
    \begin{itemize}
        \item Porty są przypisywane do odpowiednich VLAN'ów.
        \item Określane jako statyczne, port-centric lub port-based.
        Najczęstszy.
        \item Umożliwia łatwą współpracę z serwerami DHCP.
    \end{itemize}
    \item \textbf{Adresy MAC.}
    \begin{itemize}
        \item Urządzenie o danym adresie MAC będzie zawsze przypisywane do
        jednej określonej sieci VLAN bez względu na to, do którego przełącznika
        jest w danej chwili bezpośrednio dołączone.

        \item Wymaga sporo dodatkowej pracy administracyjnej.
    \end{itemize}
\end{itemize}

\subsubsection*{Korzyści}
\begin{itemize}
    \item Można łatwo zmieniać położenie stacji roboczych w sieci VLAN.
    \item Można łatwo zmieniać konfigurację sieci lokalnej VLAN.
    \item Można łatwo kontrolować ruch w sieci \parit{np. na routerach
    łączących różne sieci VLAN}.
    \item Można zwiększyć bezpieczeństo \parit{rozdzielenie użytkowników}.
\end{itemize}


\subsection*{VLAN Trunking Protocol}
\begin{itemize}
    \item \textbf{Trunk} - połączenie między przełącznikami.

    \item Połączenie typu trunk umożliwia przekazywanie ramek należących do
    różnych sieci VLAN \textbf{po jednym fizycznym nośniku}.

    \item \textbf{VTP} odpowiada za kontrolę połączenia trunk
    oraz umożliwia zautomatyzowaną konfigurację wielu switchy z jednego miejsca.

    \item Trunk może być tworzone między switchem a routerem
    \parit{\textbf{router on a stick}}.

    \item Umożliwia konfigurację wielu przełączników na drodze wymiany
    odpowiednich ramek z sąsiadującymi przełącznikami.
\end{itemize}

\subsubsection*{Frame tagging}
\begin{itemize}
    \item Z ramkami przekazywanymi połączeniem trunk wędruje informacja o
    przypisaniu ramki do odpowiedniej sieci wirtualnej.
    \item Może być realizowany na różne sposoby, np:
    \begin{itemize}
        \item Najczęściej stosowany standard IEEE 802.1Q, w którym
        \textbf{zmieniony jest nagłówek ramki 802.3}.

        \item ISL (Cisco Inter-swirch Link), \parit{przestarzała} metoda,
        w której \textbf{ramka jest enkapsulowana w ramce ISL}.
        W nagłówku ISL jest pole identyfikujące VLAN.
    \end{itemize}
\end{itemize}

\subsection*{Konfiguracja}
\begin{itemize}
    \item Konfiguracja routera dla odpowiedniej sieci VLAN jest realizowana
    poprzez \textbf{logiczne podinterfejsy \parit{sub-interfaces}}.

    \item Jeden fizyczny interfejs jest dzielony logicznie na wiele
    podinterfejsów.

    \item Router jest połączony tylko z jednym połączeniem trunk z jednym
    ze switchy, ale działa jakby było kilka interfejsów, każdy przypisany
    do innej sieci VLAN.

    \item Przełącznik można skonfigurować jako serwer VTP,
    klient VTP lub „transparent VTP”.
    Można tworzyć tzw. dziedziny składające się z wielu
    przełączników (każda dziedzina jest identyfikowana przez nazwę).
    Informacje VTP są przekazywane i interpretowane w ramach odpowiednich
    dziedzin.
    W dziedzinie jeden przełącznik powinien być serwerem, reszta to na ogół
    klienci.
    Zmiany konfiguracji przeprowadza się na serwerze.
    Dla dziedziny należy określić hasło (w celu zabezpieczenia).
\end{itemize}

\subsection*{Routowanie}

\begin{itemize}
    \item Musi być zapewnione między sieciami VLAN.
    \item Osiągane przez \textbf{router on a stick} -
    łącząc router łączem trunkingowym z pewnym przełącznikiem.
    \item Jeśli router jest wykorzystywany jedynie do przekazywania pakietów
    między różnymi sieciami VLAN, to można rozważyć też zastosowanie innego
    rozwiązania – \textbf{przełączników warstwy trzeciej}.
    \begin{itemize}
        \item  Umożliwia zdefiniowanie sieci VLAN oraz sam zapewnia routing
        między sieciami.
        \item Przełącznik ma „w sobie” router.
        \item Router ten może być łączony z innymi routerami, można na nim
        konfigurować protokoły routowania.
        \item Zwykle jednak przełącznik warstwy trzeciej nie ma interfejsów
        technologii sieci rozległych WAN.
    \end{itemize}
\end{itemize}

\pagebreak
\end{document}
