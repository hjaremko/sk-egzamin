\documentclass[../sk-egzamin.tex]{subfiles}

\begin{document}

\question{
Proszę opisać najważniejsze procesy działające w sieci komputerowej od
wpisania w przeglądarce adresu WWW do wyświetlenia strony.
}

\textbf{Komputer źródłowy:} \texttt{K1 (IP1, MAC1)}\\
\textbf{Komputer docelowy:} \texttt{K2 (IP2, MAC2, WWW2)}

Jeśli ktoś na \texttt{K1} próbuje otworzyć \texttt{WWW2}, to:
\begin{enumerate}
    \item Zadziała system DNS: \texttt{K1} skontaktuje się ze swoim
    serwerem DNS i zapyta jaki jest adres IP komputera
    związanego z nazwą domenową \texttt{WWW2}.
    Serwer DNS znajdzie odpowiedni adres w swoich zasobach i odeśle informację
    do \texttt{K1}.

    \item Przeglądarka utworzy komunikat \parit{według protokołu HTTP}.
    Do komunikatu zostanie dołączony nagłówek \parit{według protokołu TCP},
    który zawiera m. in. port docelowy \parit{standardowo 80} oraz port
    źródłowy.
    Komunikat razem z dołączonym nagłówkiem TCP nazywa się \textbf{segmentem
    TCP}.

    \item Do segmentu TCP zostanie dodany nagłówek IP - powstaje
    \textbf{pakiet IP}.

    \item Pakiet musi być przesłany w ramce.
    Do pakietu musi zostać dodany nagłówek ramki, zawierający źródłowy i
    docelowy MAC.\\
    \textbf{\texttt{K1} nie zna adresu \texttt{MAC2}}.
    Zna tylko \texttt{IP2}.
    Wykorzysytwany jest \textbf{protokół ARP}.
    \begin{itemize}
        \item \texttt{K1} wysyła \texttt{ARP Request}
        \parit{nie zawiera pakietu IP}, której adres docelowy to adres
        rozgłoszeniowy.

        \item Każdy komputer przyłączony do switcha ma obowiązek odebrać
        ramkę wysyłaną na adres rozgłoszeniowy MAC.
        Jednak tylko komputer o zadanym IP odpowie na \texttt{ARP Request}.
    \end{itemize}

    \item Odpowiedź \texttt{ARP Reply} jest wysyłana na adres \texttt{MAC1}.

    \item Po tym, jak \texttt{K1} pozna adres \texttt{MAC2} może już zbudować
    ramkę przeznaczoną do \texttt{K2}.
    Ramka jest wysłana do switcha, a switch dostarcza ją tylko do \texttt{K2}.

    \item \texttt{K2} Odbiera ramkę, sprawdza MAC docelowy i sumę kontrolną,
    po czym wyjmuje z ramki pakiet IP.
    Sprawdza adres docelowy IP i wyjmuje z pakietu segment TCP.
    Sprawdza do którego portu należy przekazać zawartość \parit{komunikat HTTP}
    i ostatecznie wyjmuje komunikat HTTP z segmentu i przekazuje do portu 80,
    na którym nasłuchuje serwer WWW.

    \item Serwer WWW konstruuje odpowiedź - stronę WWW.
    Strona ta zostaje umieczona w komunikacie HTTP, który natępnie musi być
    przesłany do \texttt{K1}. Mechanizm jest \textbf{analogiczny} jak
    poprzednio.
\end{enumerate}

\pagebreak
\end{document}
