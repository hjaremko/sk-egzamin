\documentclass[../sk-egzamin.tex]{subfiles}

\begin{document}

\question{
Co to jest \textit{ramka danych}? Proszę opisać, z czego się składa.
Czy każda ramka zawiera pakiet IP? Jeśli nie, to proszę podać przykłady.
}

\textbf{Ramka} to porcja danych przesyłana w sieciach komputerowych, tworzna
na poziomie \textbf{dostępu do sieci} \parit{warstwa druga}.\\
Urządzenie zapewniające dostęp do nośnika przesyła pewne sygnały, które są
interpretowane jako bity.\\
Wysyłany ciąg bitów zawiera pewne informacje i może być podzielony na porcje
zwane polami.

\subsection*{Pola ramki}
\begin{itemize}
    \item Ogranicznik początku ramki \parit{ustalony wzór bitów}
    \item Adres fizyczny nadawcy \parit{źródła danych}
    \item Adres fizyczny odbiorcy \parit{miejsca docelowego}
    \item Dane
    \item Ogranicznik końca ramki \parit{sekwencja kontrolna ramki}
\end{itemize}

Ogranicznik początku ramki może być poprzedzony lub może zawierać tzw.
preambułę, która w pewnych technologiach sieciowych jest stosowana do
synchronizacji nadajnika i odbiornika. Wielkość pól określana jest w oktetach.

\textbf{Uwaga:} nie wszystkie technologie sieciowe wykorzystują ramki o podanej
strukturze.

\subsection*{Czy każda ramka zawiera pakiet IP?}
\subsubsection*{Nie, na przykład ramka \texttt{ ARP Request}.}

\pagebreak
\end{document}
