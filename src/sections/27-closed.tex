\documentclass[../sk-egzamin.tex]{subfiles}

\begin{document}

\question{
Adres \texttt{192.168.7.0} z maską \texttt{255.255.252.0} to adres
\textnormal{
\begin{enumerate}[label=\alph*)]
    \item hosta
    \item poprawny, ale maska jest niepoprawna
    \item \textbf{sieci}
    \item rozgłoszenia w podsieci lokalnej
    \item niepoprawny
\end{enumerate}
}
}

\question{
Jeśli adres IP hosta jest \texttt{192.168.14.0} z maską \texttt{255.255.248.0},
to adres sieci jest następujący:
\textnormal{
\begin{enumerate}[label=\alph*)]
    \item {192.168.7.0}
    \item {192.168.14.0}
    \item \texttt{192.168.8.0}
    \item {192.168.1.0}
    \item {192.168.248.0}
\end{enumerate}
}
}

\question{
Adres \texttt{255.255.255.255} oraz \texttt{152.16.255.255} to adresy
odpowiednio:
\textnormal{
\begin{enumerate}[label=\alph*)]
    \item obydwa są niepoprawne
    \item \textbf
    {ograniczonego rozgłoszenia i rozgłoszenia w sieci \texttt{152.16.0.0}}
    \item pierwszy jest niepoprawny, drugi to rozgłoszenie w sieci lokalnej
    \item drugi jest niepoprawny a pierwszy to rozgłoszenie w sieci lokalnej
    \item pierwszy to adres domyślnej trasy w routerze, drugi to rozgłoszenie w
    sieci lokalnej
\end{enumerate}
}
}

\question{
Który z poniższych adresów hosta jest z podsieci, w której można zaadresować
62 komputery?
\textnormal{
\begin{enumerate}[label=\alph*)]
    \item \texttt{146.5.30.72/26}
    \item {182.43.23.0/30}
    \item {163.32.10.64/62}
    \item {132.43.23.33/16}
    \item {133.34.54.255/32}
\end{enumerate}
}
}

\pagebreak
\end{document}
