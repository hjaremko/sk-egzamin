\documentclass[../sk-egzamin.tex]{subfiles}

\begin{document}

\section{
Model ISO OSI, model TCP/IP.
}

\section{
Standaryzacja w sieciach komputerowych, co to są dokumenty RFC.
}

\section{
Ethernet: sposób dostępu do nośnika, ramki.
}

\section{
Ethernet: działanie przełączników i koncentratorów (podstawy).
}

\section{
Protokół IPv4: adresacja, pola w nagłówku, fragmentacja.
}

\section{
Przesyłanie grupowe (multiemisja, multicast) w IPv4 (IGMP, IGMP-snooping,
współpraca technologii Ethernet z przesyłaniem grupowym – adresy MAC przesyłania grupowego).
}

\section{
Protokół ARP.
}

\section{
Protokół ICMP.
}

\section{
Protokół UDP: charakterystyka, nagłówek.
}

\section{
Protokół TCP: charakterystyka, mechanizmy, nagłówek.
}

\section{
Protokoły routowania typu wektor odległości: sposób działania, wady i zalety,
podstawowe parametry protokołów RIP, RIP2, IGRP, EIGRP. (M.in. pętle routowania,
zliczanie do nieskończoności, dzielony horyzont, zegary).
}

\section{
Protokoły routowania stanu łącza: sposób działania, charakterystyka protokołu
OSPF, rodzaje obszarów.
}

\section{
DNS.
}

\section{
Działanie przełączników Ethernet: tryby działania, protokół STP, sieci VLAN,
łącza trunkingowe (tj. łącza wykorzystujące porty trunk), przełączniki warstwy 3.
}

\section{
Podstawy kryptografii: szyfrowanie z kluczem symetrycznym, szyfrowanie z kluczem
publicznym i prywatnym, funkcje skrótu, podpis cyfrowy, certyfikaty.
}

\section{
Bezpieczne protokoły: SSL, TLS, IPSec
(ze szczególnym naciskiem na protokół IPSec).
}


\section{
Protokół IPv6: adresacja, nagłówki, mechanizmy, ICMPv6 (m.in. jak odnaleźć adres
MAC na podstawie adresu IPv6), mechanizmy przejścia między IPv4 i IPv6,
mobilny IP.
}

\section{
Charakterystyka protokołu BGP (w zakresie omówionym na wykładzie).
}

\section{
Podstawy programowania w interfejsie gniazd (w zakresie omówionym na wykładzie).
}

\pagebreak
\end{document}
