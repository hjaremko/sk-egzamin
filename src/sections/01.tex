\documentclass[../sk-egzamin.tex]{subfiles}

\begin{document}

\section{Charakterystyka sieci LAN, WAN. Topologie połączeń.
Komutacja obwodów vs. komutacja pakietów i komutacja komórek.}

\question{
Jak wygląda i do czego służy adres z klasy adresowej C (A, B, D).\\
Proszę podać przykład.
}

\question{
Proszę ustalić adresację IP \textit{(podpisać na rysunku)} mając do dyspozycji
sieci:
\begin{itemize}[noitemsep]
    \item \texttt{41.3.3.32/28}
    \item \texttt{101.2.2.192 255.255.255.252}
    \item \texttt{48.13.16.128/29}
\end{itemize}
}
\begin{figure}[ht!]
\centering
\begin{tikzpicture}[x=1cm,y=1cm]
    \node[client,label=above:{PC C}] (c) at (1,0) {};
    \node[client,label=above:{PC D}] (d) at (3,0) {};
    \node[client,label=below:{PC A}] (a) at (2,-3) {};

    \node[router,label=below:{\small R1}] (r1) at (4,-3) {};
    \node[router,label=below:{\small R2}] (r2) at (7.5,-3) {};

    \node[client,label=below:{PC J}] (j) at (9.5,-3) {};
    \node[client,label=below:{PC B}] (b) at (11,-3) {};

    \node[client,label=above:{PC F}] (f) at (7,0) {};
    \node[client,label=above:{PC G}] (g) at (8.5,0) {};
    \node[client,label=above:{PC H}] (h) at (10,0) {};
    \node[client,label=above:{PC E}] (e) at (11.5,0) {};

    \draw[black,line width=1pt] (0.5, -1.5) -- (4.5,-1.5);
    \draw[black,line width=1pt] (6.5, -1.5) -- (12,-1.5);

    \draw[black,line width=1pt] (c) -- (1, -1.5);
    \draw[black,line width=1pt] (d) -- (3, -1.5);
    \draw[black,line width=1pt] (a) -- (2, -1.5);

    \draw[black,line width=1pt] (j) -- (9.5, -1.5);
    \draw[black,line width=1pt] (b) -- (11, -1.5);

    \draw[black,line width=1pt] (f) -- (7, -1.5);
    \draw[black,line width=1pt] (g) -- (8.5, -1.5);
    \draw[black,line width=1pt] (h) -- (10, -1.5);
    \draw[black,line width=1pt] (e) -- (11.5, -1.5);

    \draw[black,line width=1pt] (r1.east) node[above right] {{eth1}} -- (r2.west)
    node[above left] {{eth0}};


    \draw[black,line width=1pt] (4,-1.5) -- (r1.north)
    node[above left] {{eth0}};

    \draw[black,line width=1pt] (7.5,-1.5) -- (r2.north)
    node[above right] {{eth1}};

\end{tikzpicture}
\end{figure}

\question{
Co to jest \textit{ramka danych}. Proszę opisać, z czego się składa.
Czy każda ramka zawiera pakiet IP? Jeśli nie, to proszę podać przykłady.
}

\question{
Co to jest pakiet IP? Jakie pola zawiera nagłówek pakietu IPv4? Co to jest
fragmentacja pakietów IP?
}

\question{
Proszę opisać najważniejsze procesy działające w sieci komputerowej od
wpisania w przeglądarce adresu WWW do wyświetlenia strony.
}

\question{
Co to jest segment TCP? Jakie pola zawiera nagłówek segmentu TCP?
}

\question{
Jakie są podstawowe cechy protokołu TCP?
}

\question{
Na czym polega niezawodność protokołu TCP?
}

\question{
Proszę porównać protokół TCP i UDP.
}

\question{
Co to jest routing statyczny i jak można go skonfigurować na routerach?
}

\question{
 Proszę opisać (podać przykład) jak może dojść do pętli routowania w protokole
RIP (pętla ma obejmować dwa sąsiednie routery albo więcej niż dwa routery).
}

\question{
Proszę opisać działanie protokołu OSPF.
}

\question{
Proszę opisać działanie protokołu EIGRP.
}

\question{
Jak działa ARP?
}

\question{
Czym zastąpiono protokół ARP w IPv6?
}

\question{
Co to jest multiemisja \textit{(multicast)}? Co to jest IGMP?
Jakie jest wsparcie dla przesyłania grupowego w technologii Ethernet?
}

\question{
Proszę opisać protokół IPSec.
}

\question{
Proszę opisać mechanizmy w protokole IP związane z mobilnością.
}

\question{
Proszę opisać protokół IPv6.
}

\question{
Proszę opisać, jak działa mechanizm 6to4.
}

\question{
Do czego służy protokół STP i jak działa \textit{(podstawowe zasady)}.
Jak zbudować prostą lokalną sieć komputerową na przełącznikach z wykorzystaniem
warstwy dostępu \textit{(access)}, warstwy dystrybucji \textit{(distribution)}
i warstwy rdzenia \textit{(core)}?
}

\question{
Co to są VLANy i jak można je konfigurować?
}

\question{
Proszę opisać, jak działa podpis cyfrowy.
}

\question{
Proszę opisać jak sprawdzić, czy połączenie ze stroną WWW banku jest
bezpieczne.
}

\question{
Proszę podać podstawowe zasady pisania programów z wykorzystaniem
interfejsu gniazd. Jak działa serwer iteracyjny, a jak serwer współbieżny
\textit{(działający na procesach)}? Pytanie obejmuje zagadnienia w zakresie
przedstawionym na ostatnim wykładzie.
}

\question{
Proszę krótko scharakteryzować protokół BGP \textit{(w zakresie omówionym na
wykładzie)}.
}

\question{
Adres \texttt{192.168.7.0} z maską \texttt{255.255.252.0} to adres
\begin{enumerate}[label=\alph*)]
    \item hosta
    \item poprawny, ale maska jest niepoprawna
    \item sieci
    \item rozgłoszenia w podsieci lokalnej
    \item niepoprawny
\end{enumerate}
}

\question{
Jeśli adres IP hosta jest \texttt{192.168.14.0} z maską \texttt{255.255.248.0},
to adres sieci jest następujący:
\begin{enumerate}[label=\alph*)]
    \item \texttt{192.168.7.0}
    \item \texttt{192.168.14.0}
    \item \texttt{192.168.8.0}
    \item \texttt{192.168.1.0}
    \item \texttt{192.168.248.0}
\end{enumerate}
}

\question{
Adres \texttt{255.255.255.255} oraz \texttt{152.16.255.255} to adresy
odpowiednio:
\begin{enumerate}[label=\alph*)]
    \item obydwa są niepoprawne
    \item ograniczonego rozgłoszenia i rozgłoszenia w sieci \texttt{152.16.0.0}
    \item pierwszy jest niepoprawny, drugi to rozgłoszenie w sieci lokalnej
    \item drugi jest niepoprawny a pierwszy to rozgłoszenie w sieci lokalnej
    \item pierwszy to adres domyślnej trasy w routerze, drugi to rozgłoszenie w
    sieci lokalnej
\end{enumerate}
}

\question{
Który z poniższych adresów hosta jest z podsieci, w której można zaadresować
62 komputery?
\begin{enumerate}[label=\alph*)]
    \item \texttt{146.5.30.72/26}
    \item \texttt{182.43.23.0/30}
    \item \texttt{163.32.10.64/62}
    \item \texttt{132.43.23.33/16}
    \item \texttt{133.34.54.255/32}
\end{enumerate}
}

\pagebreak
\end{document}
