\documentclass[../sk-egzamin.tex]{subfiles}

\begin{document}

\question{
Jak wygląda i do czego służy adres z klasy adresowej C (A, B, D).\\
Proszę podać przykład.
}

\subsection*{Klasy adresów sieci}

\begin{tabularx}{\textwidth}{|X|X|X|X|X|}
 \hline
\textbf{Klasa} & \textbf{Adres IP} & \textbf{Adres sieci} & \textbf{Zakres 1-go bajtu} & \textbf{Najstarsze bity} \\
\hline
A & \texttt{w.x.y.z} & \texttt{w.0.0.0} & \texttt{1 - 126} & \texttt{0} \\
\hline
B & \texttt{w.x.y.z} & \texttt{w.x.0.0} & \texttt{129 - 191} & \texttt{10} \\
\hline
C & \texttt{w.x.y.z} & \texttt{w.x.y.0} & \texttt{192 - 223} & \texttt{110} \\
\hline
D & \texttt{w.x.y.z} & nie dotyczy & \texttt{224 - 239} & \texttt{1110} \\
\hline
E & \texttt{w.x.y.z} & nie dotyczy & \texttt{240 - 255} & \texttt{11110} \\
\hline
\end{tabularx}

    Same zera i jedynki nie mogą być wykorzystane.
    Same zera w części hosta oznaczają adres sieci.
    Same jedynki to adres rozgłoszeniowy w danej sieci.

\subsection*{Adresy klasy A}
\begin{itemize}
    \item $126$ sieci, każda może zawierać $2^{24}-2$, (16 777 214) komputerów
    \item Przyznane organizacjom rządowym i wielkim instytucjom.
\end{itemize}

\subsubsection*{Przykład}
\begin{center}
\texttt{1.0.0.1, 126.0.0.1, 1.255.255.254, 6.1.2.3}
\end{center}

\subsection*{Adresy klasy B}
\begin{itemize}
    \item $(191-128+1)\cdot 256 = 16 384$ sieci, każda może zawierać $2^{16}-2$,
    (65 534) komputery
    \item Sieci średniej wielkości.
\end{itemize}

\subsubsection*{Przykład}
\begin{center}
\texttt{128.0.0.1, 191.255.255.254, 128.16.0.254, 130.1.2.3}
\end{center}

\pagebreak
\subsection*{Adresy klasy C}
\begin{itemize}
    \item $(192-223+1)\cdot 256\cdot 256 = 2 097 152$ sieci,
    każda może zawierać $2^{8}-2$, (254) komputery
    \item Dużo podsieci, miało komputerów.
\end{itemize}

\subsubsection*{Przykład}
\begin{center}
\texttt{192.0.0.1, 223.255.255.254, 198.16.0.254, 220.1.2.3}
\end{center}

\subsection*{Adresy klasy D}
\begin{itemize}
    \item Przeznaczone do \textbf{transmisji grupowych}.
\end{itemize}

\subsection*{Adresy klasy E}
\begin{itemize}
    \item Adresy zarezerwowane \textit{(nie wykorzystywane normalnie do
    transmisji pakietów)}.
\end{itemize}

\subsection*{Adres loopback}
Adres postaci \texttt{127.x.y.z} \textit{(na ogół \texttt{127.0.0.1})}
zarezerwowany jest dla tzw. pętli zwrotnej \textit{(cały ruch przesyłany na ten
adres nie wychodzi z komputera)}.

\pagebreak
\end{document}
