\documentclass[../sk-egzamin.tex]{subfiles}

\begin{document}

\question{
Proszę opisać, jak działa mechanizm 6to4.
}

\subsection*{6to4}
\begin{itemize}
    \item Co jeśli ISP udostępnia \textbf{tylko} IPv4?
    \item IPv4 jest \textbf{niekompatybilne} z IPv4!
    \item \textbf{6to4} pozwala na łączenie się sieci 6to4 \textbf{z innymi}
    sieciami 6to4 lub natywnymi sieciami IPv6 \textbf{poprzez} IPv4.
\end{itemize}

\subsection*{Postać adresu}
\begin{minted}{python}
+---+------+-----------+--------+--------------------------------+
| 3 | 13   | 32        | 16     | 64 bits                        |
+---+------+-----------+--------+--------------------------------+
|FP | TLA  | V4ADDR    | SLA ID | Interface ID                   |
|001|0x0002|           |        |                                |
+---+------+-----------+--------+--------------------------------+
\end{minted}

\begin{itemize}
    \item \texttt{V4ADDR} - Enkapsulowany
    adres IPv4 udostępniony przez ISP \parit{unikalny w skali świata,
    od dostawcy}.

    \item \texttt{SLA ID} - Jak byśmy chcieli podzielić wewnątrz firmy
    na przykład.

    \item \texttt{Interface ID} - Normalnie jak w IPv6.

\end{itemize}


\subsection*{Wyspy IPv6 w morzu IPv4}

\textbf{Chcemy wysłać pakiet z komputera z sieci A do komputera w sieci B.}
\begin{itemize}
    \item Tworzony jest pakiet IPv6 z adresem docelowym B:
    \item Prefiks jest inny więc wysyłamy do routera.
    \item Następuje \textbf{enkapsulacja} pakietu IPv6 w IPv4.
\end{itemize}

\textbf{IP źródłowy:} \texttt{192.1.2.3} \\
\textbf{IP docelowy:} \texttt{09fe:fdfc -> 9.254.253.252}

\pagebreak

\begin{minted}{py}
                          _______________________________
                         |                               |
                         | Wide Area IPv4 Network        |
                         |_______________________________|
                                /                  \
                      192.1.2.3/       9.254.253.252\
 _____________________________/_   __________________\____________
|                            /  | |                   \           |
|IPv4 Site A        ##########  | |IPv4 Site B        ##########  |
| __________________# 6to4   #_ | | __________________# 6to4   #_ |
||                  # router # || ||                  # router # ||
||IPv6 Site A       ########## || ||IPv6 Site B       ########## ||
||2002:c001:0203::/48          || ||2002:09fe:fdfc::/4           ||
||_____________________________|| ||_____________________________||
|                               | |                               |
|_______________________________| |_______________________________|
\end{minted}

\subsubsection*{Połączenie 6to4 z siecią działającą jedynie wg IPv6
następuje przez „Relay router”.}

\begin{itemize}
    \item Routery graniczące NIE mogą przekazać do sieci działającej tylko
    według IPv6 trasy bardziej szczegółowej niż \textbf{2002::/16}.
    Chodzi o to, aby nie „zaśmiecać” natywnych sieci IPv6.
    W związku z tym n\textbf{nie jest poprawna konfiguracja}, w której sieci
    6to4 są połączone z dwoma niespójnymi kawałkami sieci działającej według
    IPv4.
\end{itemize}

\begin{minted}{py}
        _______________________________      ___________________
       | Wide Area IPv4 Network        |    | Native IPv6 WAN   |
       |_______________________________|    |___________________|
              /                  \             //
    192.1.2.3/       9.254.253.252\           //2001:0600::/48
____________/_   __________________\_________//_
           /  | |                   \       //  |
  ##########  | |IPv4 Site B        ##########  |
__# 6to4   #_ | | __________________# 6to4   #_ |
  # router # || ||                  # router # ||
  ########## || ||IPv6 Site B       ########## ||
             || ||2002:09fe:fdfc::/4           ||
__Site A_____|| ||_____________________________||
              | |                               |
______________| |_______________________________|
\end{minted}


\pagebreak
\end{document}
