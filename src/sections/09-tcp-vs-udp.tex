\documentclass[../sk-egzamin.tex]{subfiles}

\begin{document}

\question{
Proszę porównać protokół TCP i UDP.
}

\begin{tabularx}{\textwidth}{X|X}
\textbf{UDP} & \textbf{TCP} \\
\hline
Bezpołączeniowy & Połączenie z wykorzystaniem trójfazowego uzgodnienia\\
\hline
Nie zapiewnia niezawodności & Zapewnia niezawodność połączenia\\
\hline
Niewielki nagłówek (8 bajtów) & Duży nagłówek (minimum 20 bajtów)\\
\hline
Brak sterowania przepływem & Sterowanie przepływem \parit{okno oferowane}\\
\hline
Mogą być przesyłane z adresem docelowym przesyłania grupowego & Chyba nie?\\
\hline
Aplikacja jest odpowiedzialna za rozmiar wysyłanego datagramu &
Dane ze strumienia są dzielone na fragmentu, które wg. TCP mają najlepszy do
przesłania rozmiar\\
% \hline
\end{tabularx}

\subsection*{Nagłówek UDP}
\begin{itemize}
    \item Numer portu źródłowego \parit{16 bitów}
    \item Numer portu docelowego \parit{16 bitów}
    \item Długość UDP (nagłówek i dane) - wypełniana opcjonalne \parit{16 bitów}
    \item Suma kontrolna \parit{16 bitów}
    \begin{itemize}
        \item Jedyny mechanizm sprawdzenia nieuszkodzenia datagramu.
        \item Opjonalna w IPv4, obowiązkowa w IPv6.
    \end{itemize}
    \item Dane, jeśli są.
\end{itemize}


\pagebreak
\end{document}
